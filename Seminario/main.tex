\documentclass{article}
\usepackage{amsmath}
\usepackage[utf8]{inputenc}
\usepackage{graphicx} % Comandos para manejar imágenes
\graphicspath{ {./images/} } % Carpeta de imágenes
\usepackage[table,xcdraw]{xcolor}
\setlength{\parskip}{2mm} % Espaciado

\usepackage[utf8]{inputenc}
\usepackage{geometry}
    \geometry{left=3cm,right=2cm,top=2cm,bottom=2cm}
%
\usepackage[spanish]{babel}
%
\usepackage[fixlanguage]{babelbib}
    \bibliographystyle{babunsrt}
%

\usepackage{floatrow}
\floatsetup[table]{style=plaintop}

\usepackage{url}

\usepackage[top=2cm, bottom=2.5cm, right=3 cm, left=3 cm]{geometry} % margenes

\usepackage{parskip} % Sangria

\title{Taller A1: Metodología de la Investigación: Indagación en artículo de investigación}
\author{Cristóbal Galleguillos Ketterer$^{1}$\\
\small{$^{1}$Industrial PhD Program}\\
\small{Pontificia Universidad Católica de Valparaíso}\\
\small{cristobal.galleguillos@pucv.cl}
}
\date{\small{\today}}

\begin{document}

\maketitle

\section{Consideraciones previas}

Durante esta investigación no se consiguió el objetivo de encontrar un trabajo de investigación con financiamiento chileno que esté dentro de las posibles lineas de investigación de este estudiante.

De forma excepcional, se presenta un trabajo de investigación que se enmarca en la linea de trabajo, y que incluye una gran base de datos, además de ser un trabajo de investigación financiado por organismos gubernamentales chinos.

\section{Problema de investigación}

El estudio plantea la necesidad de evaluar las políticas públicas relacionadas con el crecimiento económico incorporando la variable la eficiencia de la generación de energía y las emisiones de gases efecto invernadero.

\section{Marco teórico en el cual se encuentra la investigación}

El marco teórico de la investigación del documento en estudio abarca las siguientes áreas:

\begin{itemize}
    \item Crecimiento económico.
    \item Emisiones de gases efecto invernadero.
    \item Análisis matemático de funciones de producción.
    \item Microeconomía.
    \item Director de un proyecto Basal.
    \item Modelos de contaminación ambiental.
\end{itemize}

\section{Alcance de la investigación}

Este trabajo Analiza la eficiencia energética desde un punto de vista dinámico y estático para varias regiones de China.

\section{Hipótesis desarrollada en la investigación}

La hipótesis planteada por el autor es que la introducción de medidas que regulen la emisión de gases de efecto invernadero es adecuada y no afecta al crecimiento económico.

\section{Datos utilizados en la investigación}

Los datos utilizados en la investigación son:

\begin{itemize}
    \item Producto interno bruto por zona.
    \item Emisiones de gases efecto invernadero.
    \item Capital de trabajo.
    \item Número de empleos.
    \item Consumo de energía.
\end{itemize}

\section{Población y la muestra en estudio}

El estudio fue ralizado para 30 provincias de China (población) con datos de un periodo comprendido entre 1997 y 2011 (muestra). 
\nocite{*}
    \bibliography{src/LeivaA2}

\end{document}
